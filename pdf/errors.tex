%%% errors


\documentclass[a4j,11pt,dvipdfmx]{jsarticle}

\usepackage{amsmath,amssymb}
\usepackage{amsthm}
\usepackage{mathtools}
\usepackage{mathrsfs}
\usepackage[dvipdfmx]{hyperref}
\usepackage{pxjahyper}
\usepackage{tikz}
\usetikzlibrary{cd}
\usepackage{tikz-cd}
\usepackage{comment}




% amsthmによる定理環境の設定
\theoremstyle{definition}
\newtheorem{theorem}{定理}
\newtheorem*{theorem*}{定理}
\newtheorem{definition}[theorem]{定義}
\newtheorem*{definition*}{定義}
\renewcommand\proofname{\bf 証明}
\renewcommand{\thefootnote}{\arabic{footnote}}


% ハイパーリンクに色を付ける
\usepackage{xcolor}
\hypersetup{
	colorlinks=true,
	citecolor=blue,
	linkcolor=blue,
	urlcolor=blue,
}



% section番号の表示を変更する
\renewcommand{\thesection}{第\arabic{section}章}



%% 誤植表の色付け
\usepackage{colortbl}
\usepackage{ulem} %\sout で取り消し線を引く
%\textcolor{red}{   }
\newcommand{\red}[1]{\textcolor{red}{#1}} 
%%% 文字を red にする。使い方:\ref{hoge}
\newcommand{\orangefull}[1]{\multicolumn{1}{|>{\columncolor[cmyk]{0,0.3,1,0}}c|}{#1}}
%%% ページ番号のマスをオレンジにする。使い方:\ref{299}




\title{}
\author{@%
	\thanks{Twitter: \url{} }}
\date{}



\begin{document}
%	\maketitle
%\begin{abstract}
%\end{abstract}
%\tableofcontents

\begin{center}
  \Large{「ファイバー束とホモトピー」 非公式誤植表} 
  \footnote{玉木大,『ファイバー束とホモトピー』(森北出版),2020/4/30発行 第1版第1刷 準拠}

 \large{compiled on \today}
 \vspace{5mm}
\end{center}



表でページ番号のマスが\fcolorbox{black}[cmyk]{0,0.3,1,0}{オレンジ}になっているところは,気をつけるべき誤植と感じたところです.




\section{ファイバーを束ねる}


\section{雛形としての被覆空間}



{\centering
\begin{tabular}{|c|l|c|c|} \hline
 \rowcolor[gray]{0.8} 
 p & 位置 & 誤 & 正 \\ \hline
 113 & 14行目 & 射$f/s = [f,W,s] \in \mathscr{C}_S(X,Y)$に対し & 射$\red{(}f/s\red{)} = [f,W,s] \in \mathscr{C}_S(X,Y)$に対し \\ \hline
\end{tabular}}





\section{ファイバー束の基本}

{\centering
\begin{tabular}{|c|l|c|c|} \hline
	\rowcolor[gray]{0.8} 
	p & 位置 & 誤 & 正 \\ \hline
	
	 & 行目 &  &  \\ \hline
\end{tabular}}




\section{ファイバー束の分類}

{\centering
\begin{tabular}{|c|l|c|c|} \hline
	\rowcolor[gray]{0.8} 
	p & 位置 & 誤 & 正 \\ \hline
	298 & 11行目 & 任意の$k \in \mathbb{N}$に対して & \red{(1)} 任意の$k \in \mathbb{N}$に対して \\ \hline
	\orangefull{299} & 5-6行目 & \begin{tabular}{c}
		$\ell$をこれら有限個の$j$すべてより\\大きくなるようにとれば
	\end{tabular} & \begin{tabular}{c}
	$\ell \mathrel{\red{\geq k}}$をこれら有限個の$j$すべてより\\大きくなるようにとれば
	\end{tabular} \\ \hline
\end{tabular}
}



\section{ファイブレーション}

{\centering
\begin{tabular}{|c|l|c|c|} \hline
	\rowcolor[gray]{0.8} 
	p & 位置 & 誤 & 正 \\ \hline
	 & 行目 &  &  \\ \hline
\end{tabular}
}




\section{あとがきに代えて}

{\centering
\begin{tabular}{|c|l|c|c|} \hline
	\rowcolor[gray]{0.8} 
	p & 位置 & 誤 & 正 \\ \hline
	394 & 20行目 & また,(6.41)の右側の四角形 & (注1) \\ \hline
	 & 行目 &  &  \\ \hline
\end{tabular}
}
\vspace{\baselineskip}

(注1): 図式(6.41)ではない.正しくは,~




\setcounter{section}{0}
\renewcommand{\thesection}{付録}%
\renewcommand{\thesubsection}{\Alph{section}.\arabic{subsection}}

\section{その他諸々の話題}

{
\centering
\begin{tabular}{|c|l|c|c|} \hline
	\rowcolor[gray]{0.8} 
	p & 位置 & 誤 & 正 \\ \hline
	
	 & 行目 &  &  \\ \hline
\end{tabular}
}








\end{document}