% beamer.tex (templete)
% コンパイルはplatexでいい

\documentclass[dvipdfmx,11pt]{beamer}
% option: handout % 完成版のスライドを表示する
% option: notheorems % theorem環境として theorem, proof, corollary, definition, definitions, fact, example, and examples があらかじめ定義されている。これを無効化する


%%%% 基本パッケージ %%%%%
%\usepackage{hyperref} %beamerが自動で読み込んでる
\usepackage{graphicx}% 各種画像の張り込み
\usepackage{comment}

%% hyperref
\usepackage{xcolor}
\definecolor{brickred}{rgb}{0.8,0.25,0.33} % dvipsnamesオプションを付けても使える
\definecolor{yelloworange}{RGB}{255,148,0} % #FF9400
\hypersetup{
	bookmarksnumbered=true, %しおりに見出し番号を付ける
	colorlinks=true, %ハイパーリンクに色を付ける
	citecolor=blue,
	linkcolor= , % textcolor
	urlcolor=yelloworange, % or orange
}


%%%% 和文用 %%%%%
\usepackage{bxdpx-beamer}
\usepackage{pxjahyper}%pdfのしおり文字化け対策。hyperrefの後ろにいれる
\usepackage{minijs}%和文用
\renewcommand{\kanjifamilydefault}{\gtdefault}%日本語フォントをゴシックにする


%%%% スライドの見た目 %%%%%
\usetheme[progressbar=frametitle]{metropolis} %デザインの選択(省略可) % or Singapore
\metroset{block=fill} % 環境の背景をうめる
\usecolortheme{rose}
\usefonttheme{professionalfonts} %フォントテーマの選択(省略可) %数式の文字を通常の LaTeX と同じにする
%\useinnertheme{circles} %フレーム内のテーマの選択(省略可) % circlesはitemizeやenumerateのitemを丸で囲む
%\useoutertheme{infolines} %フレーム外側のテーマの選択(省略可) % infolinesは上下にスライドの情報をのせる
\setbeamertemplate{navigation symbols}{}%ナビゲーションバー非表示
\setbeamertemplate{footline}[frame number]%フレーム番号をつける
\setbeamerfont{footline}{size=\normalsize,series=\bfseries}
\setbeamercolor{footline}{fg=black,bg=black}

%% 大雑把には、fg=で文字の色を、bg=でブロック背景を設定
%タイトル色
\setbeamercolor{title}{fg=structure, bg=}
%フレームタイトル色
\setbeamercolor{frametitle}{fg=structure, bg=}
%Theorem環境の色
\definecolor{beamerblue}{HTML}{3333B2} %={rgb}{0.2,0.2,0.7}, {RGB}{51,51,178}
\colorlet{blue7}{blue!7!white}
\setbeamercolor{block title}{fg=white!10!beamerblue, bg=blue7!80!fg}


%%% 消えている文字をうっすらと表示する
%\setbeamercovered{transparent}
%%%%


%%%%% フォント基本設定 %%%%%
\usepackage[T1]{fontenc}%8bit フォント
\usepackage{textcomp}%欧文フォントの追加
\usepackage[utf8]{inputenc}%文字コードをUTF-8
\usepackage{otf}%otfパッケージ
%\usepackage{lxfonts}%数式・英文ローマン体を Lxfont にする
\renewcommand{\familydefault}{\sfdefault}
\usefonttheme[onlymath]{serif}
%%%%%%%%%%

%%%% 数式 %%%%%
\usepackage{tikz} %ドライブにdvipdfmxを入れるのを忘れないように
\usetikzlibrary{cd} %tikzcdの使用
\usetikzlibrary{arrows.meta} 
\usepackage{tikz-cd}
\usepackage{bm}%数式太字
\usepackage{mathrsfs}
\DeclareMathAlphabet{\euscr}{U}{eus}{m}{n} %Euler scriptの使用
%%%%%%%%%

%%%% 定義環境 %%%%%
\usepackage{amsmath,amssymb}
\usepackage{amsthm}%amsthmをamsmathの後ろに置かないと\qedhereの挙動がおかしくなる
\usepackage{mathtools} %\coloneqq、\vcentcolon の使用
\theoremstyle{definition}
\newtheorem{thm}{Theorem}
\newtheorem{dfn}[thm]{Definition}
\newtheorem{prop}[thm]{Proposition}
\newtheorem{lem}[thm]{Lemma}
\newtheorem{cor}[thm]{Corollary}
\newtheorem{exa}[thm]{Example}
\newtheorem{remark}[thm]{Remark}
\newtheorem{conjecture}[thm]{Conjecture}
\renewcommand{\proofname}{Proof}
%%%%%%%%%


%%%%% enumerateに関する変更%%%%%
\setbeamertemplate{enumerate item}{(\arabic{enumi})}
\setbeamertemplate{enumerate subitem}{(\alph{enumii})}

%%%%%%%%%












%%%%%%%%%%%%%%%%%%%% 演算子の定義 %%%%%%%%%%%%%%%%%%%


 
 
 
 
 


%% 強調 
\newcommand{\emphasize}[1]{\textit{\textbf{#1}}}

%%%%%%%%%%%%%%%%%%%%%%%%%%%%%%%%%%%%%%%%%%%%%%%%%%%%%%%%%%%%
%%%%%%%%%%%%%%%%%%%%%%%%%%%%%%%%%%%%%%%%%%%%%%%%%%%%%%%%%%%%
\title{\LaTeX Beamer}%[略タイトル]{タイトル}
\author{Author (from University)}%[略名前]{名前}
\date[2023/1/1]{Seminar \\ 2023/1/1}
%\institute[大阪大学]{大阪大学}
%% [..]に省略名が書ける


\begin{document}

%% 表紙
\begin{frame}[plain]\frametitle{}
\titlepage
\end{frame}


%% 目次
\begin{frame}\frametitle{Contents}
\tableofcontents
\end{frame}


%% Section: Introduction
\section{Introduction}


\begin{frame}
\frametitle{Various category theories}


Some kinds of categories:
	\begin{itemize}
		\item abelian categories,
		\item topoi,
		\item triangulated categories,
		\item model categories.
	\end{itemize}
	Other notions of categories:
	\begin{itemize}
		\item enriched categories,
		\item infinity categories (quasi-categories).
	\end{itemize}

\end{frame}

\begin{frame}
	\begin{definition}
		A dg category $\mathcal{A}$ is a $\mathsf{Ch}$-enriched category.
	\end{definition}
	
	\begin{example}
		The following categories are Grothendieck categories:
		\begin{enumerate}
			\item The category $\mathsf{Ab}$ of abelian groups,
			\item The category $\mathsf{Mod}(R)$ of right modules over a ring $R$,
			\item The category $\mathsf{Qcoh}(X)$ of quasi-coherent sheaves on a scheme $X$,
			\item The category $\mathsf{Ch}=\mathsf{ChAb}$ of cochain complexes of abelian groups.
		\end{enumerate}
	\end{example}
	
\end{frame}

\begin{frame}

\begin{theorem}[Tabuada]
		The category $\mathsf{dgCat}$ of (small) dg categories admits a model category structure whose weak equivalences are quasi-equivalences.
	\end{theorem}

\begin{alertblock}{Remark}
They are used in many field of mathematics including algebraic geometry and representation theory.

\end{alertblock}
\end{frame}


%% Section: Epilogue
\section{Epilogue}

\begin{frame}
	\frametitle{References}
	
	%\footnotesize
	\small
	\begin{enumerate}[{[1]}]
		\item Goncalo Tabuada. Une structure de catégorie de modèles de Quillen sur la catégorie des dg-catégories. C.\ R.\ Math.\ Acad.\ Sci.\ Paris 340 (2005), No.\ 1, p.\ 15-19.
		
		\item Bertrand To\"en. The homotopy theory of dg-categories and derived Morita theory. Invent.\ Math.\ 167 (2007), No.\ 3, p.\ 615-667.
		
	\end{enumerate}
	
\end{frame}



\end{document}